\documentclass[12pt,letterpaper]{article}
\usepackage[utf8]{inputenc}
\usepackage[spanish]{babel}
\usepackage{amsmath}
\usepackage{amsfonts}
\usepackage{amssymb}
\author{Andrés Limón Cruz }
\begin{document}
E1. Evalúe las siguientes expresiones y muestre el resultado.\\
$$
a)~~ \left( \frac{7}{3}\right) ^2 * 4^3 * 18 - \dfrac{6^7}{9^3-652}~~~~~~b)~~509^\frac{1}{3} -4.5^2+ \dfrac{\ln(200)}{1.5} +75^\frac{1}{2}
$$
$$
~~~~~~~c)~~\dfrac{24+4.5^3}{e^{4.4}-\log_{10}(12560)}~~~~~~~~~~~~~~~~~~~d)~~\dfrac{e^{\sqrt{3}}}{\sqrt[3]{0.02-3.1^2}}~~~~~~~~~~~~~~~~~~~~~~~~~~~~~
$$
$$
e)~~\cos \left( \frac{5 \pi}{6} \right) \sin^2 \left(\frac{7 \pi}{8} \right) + \dfrac{\tan \left( \frac{\pi}{6}\ln (8)\right)}{\sqrt{7} + 2}
~~f)~~\left( \tan(64) \cos(15) \right)^2 + \dfrac{\sin^2(37)}{\cos^2(20}
$$

E.2 Defina las varibales $a,b,c$ como: $a=-18.2$, $b=6.42$, $c=a/b$, y $d=0.5(cb+2a)$, evalúe las siguientes expresiones y muestre el resultado.
$$
a)~~d-\dfrac{a+b}{c}+\dfrac{(a+d)^2}{\sqrt{\vert abc \vert}}
~~~~~~~~~~b)~~\ln((c-d)(b-a))+ \dfrac{a+b+c+d}{a-b-c-d}
$$

E.3 Para el triángulo mostrado en la figura 1, $\alpha =72$, $\beta = 43$ y su perimetro es $p=114$mm.\\
Defina $\alpha , \beta$ y $p$ como variables, y entonces:
a) Calcule los lados del triángulo usando la ley de los senos
$$
Ley~de~los~senos: \dfrac{\sin \alpha}{a}= \dfrac{\sin \beta}{b}=\dfrac{\sin \gamma}{c}
$$
b) Calcule el radio $r$ del círculo inscrito en el triángulo usando la fórmula 
$$
r= \sqrt{\dfrac{(s-a)(s-b)(s-c)}{s}}
$$
donde $s = (a+b+c)/2$\\

E.4 Muestre que 
$$
\lim_{x \rightarrow \frac{\pi}{3}} \dfrac{\sin(x- \frac{\pi}{3})}{4 \cos^2(x)-1}
$$
Para hacer esto primero cree un vector $x$ que tenga los elementos $\pi / 3 -0.1$,$\pi / 3 - 0.01$,$\pi / 3 - 0.0001$,$\pi / 3 + 0.0001$,$\pi / 3 + 0.01$,$\pi / 3 + 0.1$.Luego, cree un nuevo vector $y$ en el cual cada elemento es determiando a partir de los elementos de $x$ por
$$
\dfrac{\sin(x- \frac{\pi}{3})}{4 \cos^2(x)-1}
$$
Compare los elementos de $y$ con el valor $\dfrac{- \sqrt{3}}{6}$ calculando el error absoluto.\\

E.5 Muestre que la suma de la serie infinita 
\[
\sum_{n=1}^{\infty} \dfrac{n^2}{2^n}
\]
converge a 6. Haga esto calculando:
\[
a)~~\sum_{n=1}^{5} \dfrac{n^2}{2^n}~~~~~b)~~\sum_{n=1}^{15} \dfrac{n^2}{2^n}~~~~~c)~~\sum_{n=1}^{30} \dfrac{n^2}{2^n}
\]
Para hacer esto, para cada inciso cree un vector $n$ en el cual el primer elemento sea 1, el incremento sea 1 y el ultimo termino sea 5,15 ó 30. Luego, use las operaciones elemento a elemento para crear un vector cuyos elementos sean $\frac{n^2}{2n}$. Finalmente, use sum para sumar los términos de la serie . Compare los valores obtenidos en los incisos a), b), c) conel valor de 6 al calcular el error absoluto.\\

E.6 Cree las siguientes tres matrices:
$$
    A = \begin{pmatrix}
        5 & -3 & 7 \\
        1 & 0 & -6 \\
        -4 & 8 & 9
    \end{pmatrix}\quad
    B = \begin{pmatrix}
        3 & 2 & -1 \\
        6 & 8 & -7 \\
        4 & 4 & 0
    \end{pmatrix}\quad
    C = \begin{pmatrix}
        -9 & 8 & 3 \\
        1 & 7 & -5 \\
        3 & 3 & 6
    \end{pmatrix}
$$
a) Calcule $A+B$ y $B+A$ para mostrar que la suma de matrices es conmutativa\\
b) Calcule $A+(B+C)$ y $(A+B)+C$ para mostrar que la suma de matrices asociativa\\
c) Calcule $5(A+C)$ y $5A+5C$ para mostrar que la multiplicación por un escalar es distributiva.\\
d) Calcule $A(B+C)$ y $AB+AC$, para mostrar que la multiplicación de matrices es distributiva.\\
e) ¿Se cumple que $AB = BA$?
\end{document}