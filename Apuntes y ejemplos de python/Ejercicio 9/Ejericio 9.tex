\documentclass[12pt,letterpaper]{report}
\usepackage[utf8]{inputenc}
\usepackage[spanish]{babel}
\usepackage{amsmath}
\usepackage{amsfonts}
\usepackage{amssymb}
\usepackage[left=2cm,right=2cm,top=2cm,bottom=2cm]{geometry}
\author{Andrés Limón Cruz }
\begin{document}
E1. El índice de masa corporal (IMC) es una medida de la obesidad y este se cálculo mediante la formula
$$
IMC = \frac{P}{A^2}
$$
donde $P$ es el peso en kilogramos y $A$ es la altura en metros. La clasificación de obesidad es:
\begin{center}
\begin{tabular}{|c|c|}
\hline 
IMC & Clasificación \\ 
\hline 
$<$18.49 & Bajo peso \\ 
\hline 
18.5 - 24.9 & Normal \\ 
\hline 
25 - 29.9 & Sobrepeso \\ 
\hline 
$>$29.9 & Obesidad \\ 
\hline 
\end{tabular} 
\end{center}

E1.1 Escriba un programa que calcule el IMC de una persona. Se debe pedir all usuario su peso en kilogramos y su altura en centímetros. Como resultado se debe desplegar un enunciado que diga "Tu IMC es VVV, cuya clasificación es CCC", donde VVV es el valor de IMC calculado, y CCC es la clasificación correspondiente\\

E1.2 Pruebe su programa calculando el IMC y obteniendo la clasificación para las personas con los siguientes datos:
\begin{center}
a) Altura: 188cm, peso: 82kg~~~~~~~~~~~~b) Altura: 155cm, peso: 68kg
\end{center}
Reporte en una celda Markdown sus resultados
\end{document}